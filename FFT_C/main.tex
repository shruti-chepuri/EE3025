
\documentclass[journal,12pt,twocolumn]{IEEEtran}
\usepackage{setspace}
\usepackage{gensymb}
\singlespacing
\usepackage[cmex10]{amsmath}

\usepackage{amsthm}

\usepackage{mathrsfs}
\usepackage{txfonts}
\usepackage{stfloats}
\usepackage{bm}
\usepackage{cite}
\usepackage{cases}
\usepackage{subfig}

\usepackage{longtable}
\usepackage{multirow}
\usepackage{algorithm}
\usepackage{algorithmic}
\usepackage{enumitem}
\usepackage{mathtools}
\usepackage{steinmetz}
\usepackage{tikz}
\usepackage{circuitikz}
\usepackage{verbatim}
\usepackage{tfrupee}
\usepackage[breaklinks=true]{hyperref}
\usepackage{graphicx}
\usepackage{tkz-euclide}

\usetikzlibrary{calc,math}
\usepackage{listings}
    \usepackage{color}                                            %%
    \usepackage{array}                                            %%
    \usepackage{longtable}                                        %%
    \usepackage{calc}                                             %%
    \usepackage{multirow}                                         %%
    \usepackage{hhline}                                           %%
    \usepackage{ifthen}                                           %%
    \usepackage{lscape}     
\usepackage{multicol}
\usepackage{chngcntr}

\DeclareMathOperator*{\Res}{Res}

\renewcommand\thesection{\arabic{section}}
\renewcommand\thesubsection{\thesection.\arabic{subsection}}
\renewcommand\thesubsubsection{\thesubsection.\arabic{subsubsection}}

\renewcommand\thesectiondis{\arabic{section}}
\renewcommand\thesubsectiondis{\thesectiondis.\arabic{subsection}}
\renewcommand\thesubsubsectiondis{\thesubsectiondis.\arabic{subsubsection}}


\hyphenation{op-tical net-works semi-conduc-tor}
\def\inputGnumericTable{}                                 %%

\lstset{
%language=C,
frame=single, 
breaklines=true,
columns=fullflexible
}
\begin{document}


\newtheorem{theorem}{Theorem}[section]
\newtheorem{problem}{Problem}
\newtheorem{proposition}{Proposition}[section]
\newtheorem{lemma}{Lemma}[section]
\newtheorem{corollary}[theorem]{Corollary}
\newtheorem{example}{Example}[section]
\newtheorem{definition}[problem]{Definition}

\newcommand{\BEQA}{\begin{eqnarray}}
\newcommand{\EEQA}{\end{eqnarray}}
\newcommand{\define}{\stackrel{\triangle}{=}}
\bibliographystyle{IEEEtran}
\raggedbottom
\setlength{\parindent}{0pt}
\providecommand{\mbf}{\mathbf}
\providecommand{\pr}[1]{\ensuremath{\Pr\left(#1\right)}}
\providecommand{\qfunc}[1]{\ensuremath{Q\left(#1\right)}}
\providecommand{\sbrak}[1]{\ensuremath{{}\left[#1\right]}}
\providecommand{\lsbrak}[1]{\ensuremath{{}\left[#1\right.}}
\providecommand{\rsbrak}[1]{\ensuremath{{}\left.#1\right]}}
\providecommand{\brak}[1]{\ensuremath{\left(#1\right)}}
\providecommand{\lbrak}[1]{\ensuremath{\left(#1\right.}}
\providecommand{\rbrak}[1]{\ensuremath{\left.#1\right)}}
\providecommand{\cbrak}[1]{\ensuremath{\left\{#1\right\}}}
\providecommand{\lcbrak}[1]{\ensuremath{\left\{#1\right.}}
\providecommand{\rcbrak}[1]{\ensuremath{\left.#1\right\}}}
\theoremstyle{remark}
\newtheorem{rem}{Remark}
\newcommand{\sgn}{\mathop{\mathrm{sgn}}}
\providecommand{\abs}[1]{\left\vert#1\right\vert}
\providecommand{\res}[1]{\Res\displaylimits_{#1}} 
\providecommand{\norm}[1]{\left\lVert#1\right\rVert}
%\providecommand{\norm}[1]{\lVert#1\rVert}
\providecommand{\mtx}[1]{\mathbf{#1}}
\providecommand{\mean}[1]{E\left[ #1 \right]}
\providecommand{\fourier}{\overset{\mathcal{F}}{ \rightleftharpoons}}
%\providecommand{\hilbert}{\overset{\mathcal{H}}{ \rightleftharpoons}}
\providecommand{\system}{\overset{\mathcal{H}}{ \longleftrightarrow}}
	%\newcommand{\solution}[2]{\textbf{Solution:}{#1}}
\newcommand{\solution}{\noindent \textbf{Solution: }}
\newcommand{\cosec}{\,\text{cosec}\,}
\providecommand{\dec}[2]{\ensuremath{\overset{#1}{\underset{#2}{\gtrless}}}}
\newcommand{\myvec}[1]{\ensuremath{\begin{pmatrix}#1\end{pmatrix}}}
\newcommand{\mydet}[1]{\ensuremath{\begin{vmatrix}#1\end{vmatrix}}}
\numberwithin{equation}{subsection}
\makeatletter
\@addtoreset{figure}{problem}
\makeatother
\let\StandardTheFigure\thefigure
\let\vec\mathbf
\renewcommand{\thefigure}{\theproblem}
\def\putbox#1#2#3{\makebox[0in][l]{\makebox[#1][l]{}\raisebox{\baselineskip}[0in][0in]{\raisebox{#2}[0in][0in]{#3}}}}
     \def\rightbox#1{\makebox[0in][r]{#1}}
     \def\centbox#1{\makebox[0in]{#1}}
     \def\topbox#1{\raisebox{-\baselineskip}[0in][0in]{#1}}
     \def\midbox#1{\raisebox{-0.5\baselineskip}[0in][0in]{#1}}
\vspace{3cm}
\title{EE3025 FFT Implementation}
\author{C Shruti - EE18BTECH11006}
\maketitle
\newpage
\renewcommand{\thefigure}{\theenumi}
\renewcommand{\thetable}{\theenumi}
\bigskip
Download all the C and Python codes from 
\begin{lstlisting}
https://github.com/shruti-chepuri/EE3025/blob/main/FFT_C/codes
\end{lstlisting}


Run the following commands to compile and then run the C files to generate .dat files
\begin{lstlisting}
gcc FILENAME.c -o FILENAME -lm
./FILENAME
\end{lstlisting}
The Discrete Fourier Transform:
\begin{align}
        X(k) \triangleq \sum_{n=0}^{N-1} x(n) e^{-j 2 \pi k n / N}, \quad k=0,1, \ldots, N-1
\end{align}
The Fast Fourier Transform(FFT) is arguably the most remarkable, practical numerical algorithm for data analysis. It is fast, and has running time in the order of  O(nlogn).\\
It uses the divide-and-conquer paradigm-divide the problem into smaller sub-problems, solve them and combine the solutions.Solve the sub problems recursively. It can be a case of dynamic programming as well.\\
The algorithm can be explained with the following pseudo code:\\

\textbf{Pseudo Code}
\begin{algorithm}
\caption{fft(x)}
\begin{algorithmic} 
\STATE $n = length (x)$ such that $n = 2^k$
\IF{$n = 1$}
\STATE Return x 
\ELSE
\STATE $y^{[0]} =FFT$ (x[0],x[2]...,x[N-2])
\STATE  $y^{[1]}=FFT$ (x[1],x[3]...,x[N-1])
\FOR{$k=0; k <= n/2-1; k++$}
\STATE $y_{k}$= $y_{k}^{[0]}$ + $e^{2\pi jk/N}$ $y_{k}^{[1]}$
\STATE $y_{k+ \frac{n}{2}}$= $y_{k}^{[0]}$ - $e^{2\pi jk/N}$ $y_{k}^{[1]}$
\ENDFOR
\STATE Return y
\ENDIF
\end{algorithmic}
\end{algorithm}

The following c code generates the FFT output for the given input={\brak{1,2,3,4,4,3,2,1}}, prints the output as well as stores it in a .dat file.
\begin{lstlisting}
https://github.com/shruti-chepuri/EE3025/blob/main/FFT_C/codes/fft.c
\end{lstlisting}
The following python code reads the data from the .dat file and plots it.
\begin{lstlisting}
https://github.com/shruti-chepuri/EE3025/blob/main/FFT_C/codes/read_dat_py.py
\end{lstlisting} 
\begin{figure}
    \centering
    \includegraphics[width=\columnwidth]{./figs/fft_c.eps}
    \caption{FFT implementation in C}
    \label{fig:fft_c}
\end{figure}
Figure.\ref{fig:fft_c} is the implementation of the fft algorithm in C and plotting the spectrum in python.\\

\textbf{Verification with python built in fft function(numpy)} \\

The following code computes the FFT of the given input in python using the in-built numpy.fft.fft function and plots the output.
\begin{lstlisting}
https://github.com/shruti-chepuri/EE3025/blob/main/FFT_C/codes/fft.py
\end{lstlisting}
\begin{figure}
    \centering
    \includegraphics[width=\columnwidth]{./figs/fft_py.eps}
    \caption{FFT implementation in Python}
    \label{fig:fft_py}
\end{figure}
Figure.\ref{fig:fft_py} is the computation and plotting of FFT spectrum in python.\\
We can see that our FFT computation matches that of the in-built function. 
\end{document}
